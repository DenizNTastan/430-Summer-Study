\chapter{The First Law of Thermodynamics}
\label{ch:Chapter1}
    Before we begin our studies on thermodynamics and statistical mechanics, we first need some fundamental definitions.
\section{Definitions}
\label{sec:1.1Definitions}
    \paragraph{System:}A collection of a large number of entities, fully or partially isolated from the rest of the universe
    \paragraph{Walls:}Boundaries between a system and the rest of the universe.
    There are three kinds of walls:
    \begin{itemize}
        \item \textbf{Adiabatic Walls:} Walls that totally isolate the system. Does not allow any passage of energy or particles.
        \item \textbf{Diathermal Walls:} Walls that only allow passage of energy.
        \item \textbf{Permeable Walls:} Walls that only allow passage of particles.
    \end{itemize}
    \paragraph{State:}Each value of a set of variables that completely describe the relevant properties of a system. 
    \begin{itemize}
        \item \textbf{Macrostates:} States with macroscopic properties that can be determined by simple measurement devices.
        \item \textbf{Microstates:} States with variables as the properties that constitute the system.
    \end{itemize}
    \begin{equation*}
        (\theta_G, V, P) \longrightarrow \text{Macrostate} \hspace{1cm} (\v{r}_1,\v{p}_1,\v{r}_2,\v{p}_2,...) \longrightarrow \text{Microstate}    
    \end{equation*}
    Here, the microstate consists of the positions and momenta of each entity that consitute the system whereas the macrostate consists of the three fundamental macroscopic properties of a system.
    \begin{itemize}
        \item \textbf{Volume} (V): The amount of space that the system occupies in the universe.
        \item \textbf{Pressure} (P): The amount of force per unit area that the particles within the system apply to the walls of the system.
        \item \textbf{Temperature} ($\theta_G$): Temperature is hard to precisely define. We leave its precise definition to later chapters. For now, let us define it as a measure of the amount of heat given to a system.
    \end{itemize}
    \paragraph{Equilibrium:}State of a system at which there is no change over time in macroscopic properties. 
    
    For a state to be a viable state of equilibrium, there are two conditions.
    \begin{enumerate}
        \item The final state that is the equilibrium must be independent of the initial state of the system.
            \begin{figure}[H]
                \centering
                \begin{tikzpicture}
    % First plot
                \begin{scope}[shift={(0,0)}]
                    \draw[->] (0,0) -- (4,0) node[right] {$V$};
                    \draw[->] (0,0) -- (0,3) node[above] {$P$};
                    \draw[dashed] (0,1.5) -- (3.5,1.5);
                    \node[left] at (0,1.5) {$P_{eq}$};
                    \draw[thick, red] (0,0.5) to[out=60,in=180] (3.5,1.5);
                    \node[below left] at (0,0.8) {$P_0$};
                \end{scope}
                
                % Second plot
                \begin{scope}[shift={(6,0)}]
                    \draw[->] (0,0) -- (4,0) node[right] {$V$};
                    \draw[->] (0,0) -- (0,3) node[above] {$P$};
                    \draw[dashed] (0,1.5) -- (3.5,1.5);
                    \node[left] at (0,1.5) {$P_{eq}$};
                    \draw[thick, red] (0,2.5)
                    to[out=-60,in=180] (3.5,1.5);
                    \node[above left] at (0,2.3) {$P_0$};
                \end{scope}
            \end{tikzpicture}
                \caption{A system at two different initial values, reaching the same equilibrium state.}
                \label{fig:equilibrium}
            \end{figure}
            
        \item The equilibrium state must be independent of the history of the system
    \end{enumerate}
    Equilibrium can be reached after given enough time in a single system as well as a collection of interacting systems. When it comes to which property stops changing over time, there are three aspects of an equilibrium state.
    \begin{enumerate}
        \item \textbf{Mechanical Equilibrium:} No change over time in the volume of the system.
        \item \textbf{Thermal Equilibrium:} No heat exchange happens between the systems.
        \item \textbf{Chemical Equilibrium:} Number of interacting particles in the system(s) do not change.
    \end{enumerate}
    For a reversible reaction \ce{A + B <=> AB},  chemical equilibrium is reached when the rate of the reaction in one direction equals the rate of the reaction in the opposite direction. At that point, reactions do not stop. However, since the rates are equal, there is no macroscopic change to the system. If all three aspects are reached, the system is said to be in a \textbf{thermodynamic equilibrium}. If two systems are in equilibrium with each other, then they are said to be \textbf{isothermal} to each other.
    \begin{law*}{Zeroth Law of Thermodynamics}
        For three interacting systems $A,B,C$; if the system $A$ is in equilibrium with the system $B$ and $B$ is in equilibrium with the system $C$, then $A$ and $C$ are also in equilibrium. 
    \end{law*}
    \noindent Now consider a reference system $A$, a gas with constant state variables $P$ and $V$ and a test system $X$. System $X$ is put in contact with $A$ for a long enough time such that they are always in equilibrium. Now, we vary the volume and the pressure of $X$ all the while keeping it isothermal to $A$. Then, we obtain a specific curve called an \textbf{isotherm.}
    \begin{figure}[H]
        \centering
        \resizebox{0.35\textwidth}{!}{%
        \begin{circuitikz}
        \tikzstyle{every node}=[font=\Large]
        \draw [line width=0.2pt, ->, >=Stealth] (11.5,18.5) -- (11.5,25);
        \draw [line width=0.2pt, ->, >=Stealth] (11.5,18.5) -- (19,18.5);
        \draw [line width=0.2pt, short, red] (12.5,24.5) .. controls (13.5,20.75) and (13.25,20) .. (18.5,19.25);
        \draw [line width=0.2pt, short, blue] (12,24.5) .. controls (12.5,21) and (12,19.5) .. (18.5,18.75);
        \draw [line width=0.2pt, short, violet] (13,24.5) .. controls (14.25,21.5) and (13.25,20.5) .. (18.5,19.75);
        \draw [line width=0.2pt, short, orange] (13.5,24.5) .. controls (14.75,21.75) and (14,21) .. (18.5,20.25);
        \node [font=\Large] at (19.25,18.5) {V};
        \node [font=\Large] at (11.5,25.5) {P};
        \node [font=\Large, blue] at (19,19) {$\theta_1$};
        \node [font=\Large, red] at (19,19.4) {$\theta_2$};
        \node [font=\Large, violet] at (19,19.8) {$\theta_3$};
        \node [font=\Large, orange] at (19,20.2) {$\theta_4$};
        \end{circuitikz}
        }%
        \caption{Isotherms for multiple systems $X_i$ that are isothermal to $A$.}
        \label{fig:isotherms}
    \end{figure}
    \noindent Repeating this with many other systems at other constant values of $P$ and $V$, we can find a relationship of the form 
    
    \begin{equation}
        \theta = F(P,V) .
        \label{eq: tempeos}
    \end{equation}
\section{Functions of State}
\label{sec:1.2Functionsofstate}
    A \textbf{function of state} is a quantity that depends on the present values of macroscopic variables. An example is that when a system is in equilibrium, the temperature is a function of state. Functions of state are described by \textbf{equations of state}. An example equation of state relating the temperature of the state to its pressure and volume is given in (\ref{eq: tempeos}). Any relevant quantity can be written as a function of state with respect to the remaining quantitites.
    \begin{equation*}
        \theta = F(P,V) \hspace{1cm} P = G(\theta,V) \hspace{1cm} V = H(\theta, P)
    \end{equation*}
    Generally, equations of states are complicated mathematical expressions. An exception to this is the equation of state for an ideal gas. 

    \noindent Isotherms of an ideal gas are given by $PV=$ constant. Then, we can choose the temperature of the ideal gas ($\theta_G$) to be proportional to $PV$. Let $R$ be the constant of proportionality per mole of a gas called the \textbf{gas constant}. Then, we obtain the equation of state (\textit{Ideal gas scale of temperature}).
    
    \begin{equation}
        \theta_G = \frac{PV}{R}
    \end{equation}
    
    For $n$ moles of particles, we obtain the familiar ideal gas equation.
    
    \begin{equation}
        PV=nR\theta_G
        \label{eq:idealgas}
    \end{equation} 
    
    A more complicated example of an equation of state is the equation for a Van der Waals gas:
    
    \begin{equation}
        \theta R = \lrp{P + \frac{a}{V^2}}(V-b).
        \label{eq:vdw}
    \end{equation}
    
\subsection{Mathematical Properties of Functions of States}
\label{subsec:1.2.1mathematicalproperties}
    Let us consider an equation of state given as $G=g(x,y)$. Then, the change in $G$ is given by
    
    \begin{equation}
        \d G = \periv{g}{x}dx + \periv{g}{y}dy .
    \end{equation}
    If we define $A \coloneqq \del_xg$ and $B \coloneqq \del_yg$, we get
    \begin{equation}
        \d G = A(x,y)dx + B(x,y)dy.
    \end{equation}
    
    Now consider the second derivatives of $g$. 
    
    \begin{equation}
        \periv{A}{y} = \periv{^2g}{y\del x} \hspace{0.5cm};\hspace{0.5cm} \periv{B}{x} = \periv{^2g}{x\del y}
    \end{equation}
    
    For continuous and differentiable functions, differentiation operator commutes. Then, the order of differentiation does not matter and we get
    
    \begin{equation}
        \periv{A}{y} = \periv{B}{x}.
        \label{eq:analytic}
    \end{equation}
    
    Any function $g(x,y)$ that satisfies (\ref{eq:analytic}) is called an \textbf{analytic function}. $G$ is a function of state only if $g(x,y)$ is analytic. 
    
    Now, let us obtain $g(x,y)$ given a differential $\d G$. Defining a vector $\d\v{G}$ from a vector $\v{G}(x,y)$ where
    
    \begin{equation}
        \v{G} = A\bas{x} + B\bas{y},
    \end{equation}
    we get
    \begin{equation}
        \d\v{G} = \v{G}\cdot \d\v{r}.
    \end{equation}
    If we consider a line integral around a closed contour $\gamma$
    \begin{equation}
        \oint_\gamma \v{G}\cdot\d\v{r},
    \end{equation}
    by Stokes' theorem, we get
    \begin{equation}
        \oint_\gamma \v{G}\cdot\d\v{r} = \iint_S\lrp{\nabla\times\v{G}}\cdot\d\v{S}.
    \end{equation}
    Calculating the curl, $x$ and $y$ components vanish and we get
    \begin{equation}
        \nabla\times\v{G} = (\del_xB-\del_yA)\bas{k}.
    \end{equation}
    
    If we assume that $G$ is a function of state, these two derivatives are equal to each other. Thus, we get
    
    \begin{equation}
        \oint_\gamma \v{G}\cdot\d\v{r} = 0.
        \label{eq:closecontour}
    \end{equation}
    
    For a better analysis, let us break down the integral into two paths. 
        \begin{figure}[H]
            \centering
            \resizebox{0.4\textwidth}{!}{%
            \begin{circuitikz}
            \tikzstyle{every node}=[font=\large]
            \draw [line width = 1.2pt, -, postaction={decorate}, decoration={markings, mark=at position 0.4 with {\arrow{>}}}] (12.75,21) .. controls (20.5,26.25) and (16.75,19.75) .. (23.25,21.5);
            \draw [line width=1.2pt, -, postaction={decorate}, decoration={markings, mark=at position 0.3 with {\arrow{<}}}] (12.75,21) .. controls (14.75,16.75) and (19.75,19.25) .. (23.25,21.5);
            \draw [line width=2pt, ->, >=Stealth, blue] (11,15.75) -- (22.75,21.25);
            \draw [line width=2pt, ->, >=Stealth, blue] (11,15.75) -- (12.75,20.75);
            \draw [ fill={rgb,255:red,0; green,0; blue,0} , line width=0.2pt ] (23.2,21.5) circle (0.1cm);
            \draw [ fill={rgb,255:red,0; green,0; blue,0} , line width=0.2pt ] (12.75,21) circle (0.1cm);
            \node [font=\huge, blue] at (12,20.25) {$\v{r}_1$};
            \node [font=\huge, blue] at (23,20.5) {$\v{r}_2$};
            \node [font=\huge] at (12,22) {$\gamma$};
            \node [font=\huge] at (14,18.5) {$\gamma_1$};
            \node [font=\huge] at (17,24) {$\gamma_2$};
            \end{circuitikz}}
            \caption{\label{fig:path} A closed contour $\gamma$ is broken down into two paths $\gamma_1$ and $\gamma_2$}
        \end{figure}
    \noindent Then, the integral in (\ref{eq:closecontour}) is written as
        
        \begin{equation}
             \oint_\gamma \v{G}\cdot\d\v{r} = \lrp{\int_{\v{r}_1}^{\v{r}_2}\v{G}\cdot\d\v{r}}_{\gamma_1} + \lrp{\int_{\v{r}_2}^{\v{r}_1}\v{G}\cdot\d\v{r}}_{\gamma_2} = 0.
        \end{equation}
        \begin{equation}
           \Rightarrow     \lrp{\int_{\v{r}_1}^{\v{r}_2}\v{G}\cdot\d\v{r}}_{\gamma_1} =  \lrp{\int_{\v{r}_1}^{\v{r}_2}\v{G}\cdot\d\v{r}}_{\gamma_2}
        \end{equation}     
        
    Therefore, the line integral of $\v{G}$ is independent of the path taken. This results in the integral
    
    \begin{equation}
        g(x,y) = \int_{\v{r}_1}^{\v{r}}\v{G}\cdot\d\v{r}
    \end{equation}
    
    having a well-defined and unique value for any arbitrary $\v{r} = x\bas{x}+y\bas{y}$. Now, consider a small change $\d G$. 
    \begin{equation}
        \d G = g(x+\d x, y+\d y) - g(x,y)= \int_{\v{r}_1}^{\v{r}+\d\v{r}}\v{G}\cdot\d\v{r}-\int_{\v{r}_1}^\v{r}\v{G}\cdot\d\v{r} = \int_\v{r}^{\v{r}+\d\v{r}}\v{G}\cdot\d\v{r}
    \end{equation}
    Since this integral is independent of the path between $\v{r}$ and $\v{r}+\d\v{r}$, $\d G$ is called an \textbf{exact differential}. If $A$ and $B$ do not satisfy (\ref{eq:analytic}), then $\d G \longmapsto \dbar G$ is an \textbf{inexact differential}
\newpage

\section{Internal Energy}
\label{sec:1.3internalenergy}
    Now, let us make two more definitions:
    \begin{itemize}
        \item \textbf{Process:} A change in the state of the system that takes it from an initial configuration to a final configuration.
        \item \textbf{Initial Energy:} The energy (kinetic and potential) contained within the system.
    \end{itemize}
    
    The initial energy is always defined with respect to a reference point.
    
    \begin{equation}
        U = U_0 + \Delta U
    \end{equation}
    
    The state of an isolated system, i.e., the internal is changed by either doing work on the system or heating the system. 
    Assume that the state of the system is changed by doing work. The amount of work done to cause an equal change is independent of the means by which the work is performed. Moreover if several processes occur, the final equilibrium state depends only on the total amount of work done and not on the order in which it was done. \\
    \\
    \noindent Here, we give the First Law of Thermodynamics.
    \begin{law*}{The First Law of Thermodynamics}
        In a closed system, the change in the internal energy of the system is equal to the sum of the work done and the heat applied to the system by the surroundings. When heat is taken into account energy is always conserved.
        
        \begin{equation}
            \Delta U = W+Q
        \end{equation}
    \end{law*}
    \paragraph{Note!} In chemistry, the work done is taken to be a negative quantity such that $-W > 0$. Then, the first law is written as $\Delta U = Q - W$. As we will show later, work is not an exact differential. Therefore, from the First Law, the heat is also an inexact differential. 
    
    There are two ways to change the state of a system.

    \subsection{Reversible and Irreversible Processes}
        
        Imagine a gas in an adiabatic container with a movable piston of area $A$ and a mass $m$ on the piston as given in Figure (\ref{fig:reversible}). After the system reaches equilibrium, an infinitesimal mass $\delta m$ is added. 
        
        \begin{minipage}{0.5\textwidth}
        \begin{figure}[H]
            \centering
            \resizebox{0.3\textwidth}{!}{%
            \begin{circuitikz}
            \tikzstyle{every node}=[font=\LARGE]
            \draw [line width=2pt, short] (4.25,11.25) -- (4.25,7);
            \draw [line width=2pt, short] (4.25,7) -- (8.5,7);
            \draw [line width=2pt, short] (8.5,7) -- (8.5,11.25);
            \draw [line width=2pt, short] (4.25,11.25) -- (5,11.25);
            \draw [line width=2pt, short] (5,11.25) -- (5,7.75);
            \draw [line width=2pt, short] (5,7.75) -- (7.75,7.75);
            \draw [line width=2pt, short] (7.75,7.75) -- (7.75,11.25);
            \draw [line width=2pt, short] (7.75,11.25) -- (8.5,11.25);
            \draw [line width=1pt, short] (5,10.5) -- (7.75,10.5);
            \draw [line width=1pt, short] (5,10) -- (7.75,10);
            \draw [, line width=1pt ] (5.5,11.5) rectangle (6.5,10.5);
            \node at (5.75,9.25) [circ, blue] {};
            \node at (5.5,8.25) [circ, blue] {};
            \node at (6.5,9.25) [circ, blue] {};
            \node at (7.25,9.5) [circ, blue] {};
            \node at (5.75,9.75) [circ, blue] {};
            \node at (7.25,8.75) [circ, blue] {};
            \node at (6,8.75) [circ, blue] {};
            \node at (7.25,8.25) [circ, blue] {};
            \node [font=\Large] at (6,11) {m};
            \node [font=\Large] at (7.1,10.90) {A};
            \node [font= \Large] at (6.4, 8) {P,V};
            \end{circuitikz}
            }%

            \caption{An adiabatic system at equilibrium.}
            \label{fig:reversible}
        \end{figure}
        \end{minipage}
        \hfill
        \begin{minipage}{0.5\textwidth}
            \begin{equation}
                P_0 \longmapsto P = P_0 +\delta P
            \end{equation}
            \begin{equation}
                V_0 \longmapsto V = V_0 +\delta V
            \end{equation}
            \vfill
        \end{minipage}\\
        
        After the additional mass is added, enough time passes so that the system is at an equilibrium again. At that point the mass is removed again and system is left to reach the equilibrium. Then, in total, we have
        
        \begin{equation*}
            (P_0,V_0,\theta_G^0) \xlongrightarrow[]{+\delta m} (P,V,\theta_G) \xlongrightarrow[]{-\delta m} (P_0,V_0,\theta_G^0).
        \end{equation*}
        
        The question to ask here is that how much work is done on the system by the added mass? Since the mass does gravitational work on the system, we have
        
        \begin{equation}
            \dbar W = (m+\delta m)gdz.
        \end{equation}
        
        If we divide and multiply the right-hand side by ($-A$), we get
        
        \begin{equation}
            \dbar W = -\frac{m+\delta m}{A}g(-Adz).
        \end{equation}
        
        We can now define an volume element $dV = \v{A}\cdot d\v{z} = (a\bas{k})\cdot(-dz\bas{k})=-Adz$. Substituting this into (1.3.6),
        
        \begin{equation}
            \dbar W = -\frac{m+\delta m}{A}gdV = -\frac{mg}{A}dV - \frac{\delta mg}{A}dV = -(P+\delta P)dV.
        \end{equation}
        
        Neglecting the infinitesimal term, we obtain the expression for the work done.
        
        \begin{equation}
            \dbar W = -PdV
        \end{equation}
        
        If this equality holds, then the process is said to be a \textbf{reversible process}. Therefore, the First Law for a reversible process becomes $dU = \dbar Q_{\text{rev}} -PdV$. However, since the system is insulated, $\dbar Q_\text{rev}=0$.
        
        \begin{equation}
            dU = -PdV \Rightarrow \Delta W = -\int_{V_1}^{V_2}PdV
            \label{eq:work}
        \end{equation}
        
        Now consider two point $(P_1, V_1)$ and $(P_2, V_2)$ and choose two paths connecting the points. Since the work done is the integral of the area under each path, two different paths will give two different values for the work done. Therefore, the work is not a path-independent quantity:
        
        \begin{equation}
            \oint dW \neq 0.
        \end{equation}
        
        Because of this, work is not a function of state. By the First Law, this also forces the heat to not be a function of state.
        
        For an ideal gas, (\ref{eq:work}) gives
        
        \begin{equation}
            \Delta W = -nR\theta_G\int_{V_1}^{V_2}\frac{dV}{V} = -nR\theta_G\ln\lrp{\frac{V_2}{V_1}}.
        \end{equation}
      
      In general, however, work is larger than $-PdV$. Suppose that we put a mass $\Delta m$ on the system given in Figure (\ref{fig:reversible}) that is comparable to $m$. Then, the second term in (1.3.7) cannot be neglected anymore and we get $\dbar W \geq -PdV$. Such processes are called \textbf{irreversible processes}. A common example for irreversible processes are the sudden processes.
       
        Assume that in Figure (\ref{fig:reversible}), we suddenly push the piston such that $V_0\mapsto V_0/2$ rather than adding a mass. Then the system goes into a state $(P_1, V_0/2, \theta_1)$. Since we give energy to the system, $\theta_0<\theta_1$ and since $\Delta U = \Delta W$, $P_1\neq2P_0$. After the equilibrium is reached, we suddenly pull the piston. Here, at the instant the piston is pulled, the gas molecules have not diffused to fill the space and therefore piston does not have contact with the gas molecules. Then $\Delta\theta=0\Rightarrow\Delta U = 0$. The temperature does not change and we know that $\theta_1>\theta_0$. This means that $P_2\neq P_0$. Therefore, after doing the opposite of the action, we have not reached the same state as before and as such, sudden processes are irreversible.
    \subsection{Heat Capacities}
    \label{subsec:1.3.1Heatcapacities}
        \paragraph{Heat Capacity ($C$):} A measure of how much heat is required for a given change in temperature.
        
        Since heat is not a function of state, the path must be specified. Then, we have two possibilities:
        \begin{enumerate}
            \item \textbf{Constant Volume:} From First Law, $dU = \dbar Q + \dbar W$.
            
            \begin{equation}
                \Rightarrow C_V = \periv{Q}{\theta} = \periv{U}{\theta}-\periv{W}{\theta}
            \end{equation}
            
            For reversible processes $\dbar W = -pdV$ and since volume is constant, $dV=0$.
            
            \begin{equation}
                \therefore C_V = \periv{U}{\theta}
            \end{equation}
            
            \item \textbf{Constant Pressure:} 
            
            \begin{equation}
                C_P = \periv{Q}{\theta} = \periv{U}{\theta}-\periv{W}{\theta} = C_V+P\periv{V}{\theta}
            \end{equation}
        \end{enumerate}
        In the second case, energy goes to change both the temperature and the volume. Thus, we can see that $C_P > C_V$. For an ideal gas; $PV = nR\theta$ and $\boxed{U=\frac{3}{2}nR\theta}$\footnote{This formula will be discussed later.}. 
        \begin{equation}
            \boxed{C_V = \periv{U}{\theta} = \frac{3nR}{2} \hspace{0.5cm},\hspace{0.5cm} C_P = C_V + P\periv{V}{\theta}=\frac{3}{2}nR + P\frac{nR}{P} = \frac{5}{2}nR}
        \end{equation}
        
        \begin{problem}{Derive pressure as a function of volume in a reversible, \\adiabatic process for an ideal gas.}
            kSince the process is adiabatic, $dU = \dbar Q + \dbar W = -PdV$.
            On the other hand, $dU = \periv{U}{\theta}d\theta = C_Vd\theta$. Combining these two, we get $d\theta = - \frac{P}{C_V}dV$. From ideal gas equation,
            \begin{equation}
                PV = nR\theta \Rightarrow PdV + VdP = nRd\theta = -\frac{nRP}{C_V}dV.
            \end{equation}
            \begin{equation}
                PdV\lrp{1+\frac{nR}{C_V}} + VdP = 0
            \end{equation}
            The term in the parenthesis can be written as $(C_V+nR)/C_V = C_P/C_V$. Defining $\gamma = C_P/C_V$, we get
            \begin{equation}
                P\gamma dV + VdP = 0.
            \end{equation}
            Dividing both sides by $PV$ and integrating yields,
            \begin{equation}
                \ln P + \gamma\ln V = C' \Rightarrow PV^\gamma = C.
            \end{equation}
            Then we obtain,
            \begin{equation}
                \boxed{P = CV^{-\gamma}}
            \end{equation}
        \end{problem}
    In a more general case, the equilibrium state is characterised by $U$ and a general set of variables $\v{X}$. Then the First Law is $dU = \dbar Q + \v{f}\cdot d\v{X}$ where $\v{f}$ is a vector of generalised coordinates and $d\v{x}$ corresponds to a set of conjugate variables. We will see that we can write the heat using the temperature as a variable as $\dbar Q = TdS$. Then, $U$ becomes a function $U=U(S,\v{X})$. Similarly, we can incorporate $T$ into $\v{f}$. Note that $\v{f}$ are intersive (size-independent) variables while $\v{x}$ are extensive variables. 

    Depending on the experimental conditions, it may be necessary to switch between these two sets. This is done by defining a function $f=f(x_1,...,x_n)$. Then, $df = \sum_{i=1}^nu_idx_i$ where $u_i$ is the partial derivative of $f$ with respect to $x_i$. At this point, we can define another function that switches the focus from $x_i$ to $u_i$ for some of the variables.
        \begin{equation}
            g \coloneqq f-\sum_{i=r+1}^nu_ix_i \Rightarrow dg = df-\sum_{i=r+1}^n(u_idx_i+x_idu_i)
        \end{equation}
        Substituting (1.3.21),
        \begin{equation}
            dg = \sum_{i=1}^n u_idx_i - \sum_{i=r+1}^n(u_idx_i+x_idu_i) = \sum_{i=1}^ru_idx_i-\sum_{i=r+1}^nx_idu_i.
        \end{equation}
        Thus, we have obtained $g$ as $g=g(x_1,...x_r;u_{r+1},...,u_n)$. This function that connects $x_i$ to $u_i$ defined by the function $f$ is called the \textbf{Legendre Transformation} of $f$.
\section{Thermodynamic Functions}
\label{sec:1.4thermodynamicfunctions}
        The equilibrium state is a stationary point in the "energy" function. The particular function to be minimised depends on the experimental conditions. In general, we have three such functions.
        \begin{enumerate}
            \item \textbf{Helmholtz Free Energy (F)}: $S\longleftrightarrow \theta$, $F=U-\theta S$.
            \begin{equation}
                dF = dU -Sd\theta-\theta dS = \theta dS -PdV -Sd\theta-\theta dS=-(PdV+Sd\theta)
            \end{equation}
            \item \textbf{Gibbs Free Energy (G)}: $S\longleftrightarrow \theta$, $P\longleftrightarrow-V$, $G=U-\theta S+PV$.
            \begin{equation}
                dG = \theta dS-PdV-\theta dS-Sd\theta+PdV+VdP = VdP-Sd\theta
            \end{equation}
            \item \textbf{Enthalpy (H):} $P\longleftrightarrow-V$, $H=U+PV$.
            \begin{equation}
                dH = \theta dS - PdV + PdV + VdP = \theta dS+VdP
            \end{equation}
        \end{enumerate}
        Looking at the derivative of enthalpy with respect to temperature,
        \begin{equation}
            \periv{H}{\theta} = \periv{U}{\theta} + P\periv{V}{\theta} = \periv{Q}{\theta}-P\periv{V}{\theta}+P\periv{V}{\theta}=\periv{Q}{\theta}
        \end{equation}
        \begin{equation}
            \therefore C_P = \periv{H}{\theta}
        \end{equation}
    \section{Exercises}
        \begin{eocproblem*}{1.2 from Bowley \& Sanchez}
            By writing the internal energy as a function of state $U(\theta,V)$, show that 
            \begin{equation}
                \dbar Q = \lrp{\periv{U}{\theta}}_Vd\theta+\lrb{\lrp{\periv{U}{V}}_\theta+P}dV.
            \end{equation}
        \end{eocproblem*}
        \noindent We know that we have, 
        \begin{equation}
            dU = \dbar Q + \dbar W = \dbar Q -PdV.
        \end{equation}
        Moreover since $U=U(\theta,V)$,
        \begin{equation}
            dU = \periv{U}{\theta}d\theta + \periv{U}{V}dV.
        \end{equation}
        Then,
        \begin{equation}
            \boxed{\dbar Q = \periv{U}{\theta}d\theta + \lrp{\periv{U}{V}+P}dV}.
        \end{equation}
\newpage
        \begin{eocproblem*}{1.4 from Bowley \& Sanchez}
            The equations listed below are not exact differentials. Find for each equation an integrating factor, $g(y,z) = y^\alpha z^\beta$, where $\alpha$ and $\beta$ can be any number, that will turn it into an exact differential.\\
            (a) $dx = 12z^2dy+18yzdz$\\
            (b) $dx = 2e^{-z}dy-ye^{-z}dz$
        \end{eocproblem*}
        \textbf{A]} 
        \begin{equation}
            dx = 12z^2dy + 18yzdz \Rightarrow  dx' = 12z^2y^\alpha z^\beta dy + 18yzy^\alpha z^\beta dz
        \end{equation}
        \begin{equation}
            \periv{x}{y} = 12y^\alpha z^{2+\beta} \hspace{0.5cm} \& \hspace{0.5cm} \periv{x}{z} = 18y^{\alpha+1}z^{\beta+1}
        \end{equation}
        \begin{equation}
            \begin{cases}
                \periv{^2x}{z\del y} = 12(2+\beta)y^\alpha z^{\beta+1} \\ 
                \periv{^2x}{y\del z} = 18(\alpha + 1)y^\alpha z^{\beta+1}
            \end{cases} \Rightarrow 12(2+\beta)y^\alpha z^{\beta+1} = 18(\alpha+1)y^\alpha z^{\beta+1}
        \end{equation}
        \begin{equation}
            \Rightarrow 12(2+\beta) = 18(\alpha+1) \Rightarrow 12\beta = 18\alpha -6 \Rightarrow 2\beta = 3\alpha-1
        \end{equation}
        \begin{equation}
            \boxed{\therefore \beta = \frac{1}{2}(3\alpha-1)}
        \end{equation}
        \textbf{B]}
        \begin{equation}
            dx = 2e^{-z}dy - ye^{-z}dz \Rightarrow dx' = 2e^{-z}y^\alpha z^\beta dy - ye^{-z}y^\alpha z^\beta dz
        \end{equation}
        \begin{equation}
            \begin{cases}
                \periv{^2x}{z\del y} = -2e^{-z}y^\alpha z^\beta +2\beta e^{-z}y^\alpha z^\beta \\ 
                \periv{^2x}{y\del z} = -e^{-z}(\alpha+1)y^\alpha z^\beta
            \end{cases} 
        \end{equation}
        \begin{equation}
            \Rightarrow -2e^{-z}y^\alpha z^\beta +2\beta e^{-z}y^\alpha z^\beta = -e^{-z}(\alpha+1)y^\alpha z^\beta \Rightarrow 2(\beta-1) = - (\alpha + 1)
        \end{equation}
        \begin{equation}
            \Rightarrow 2\beta-2 = -\alpha-1 \Rightarrow 2\beta = 1-\alpha
        \end{equation}
        \begin{equation}
            \boxed{\therefore \beta = \frac{1}{2}(1-\alpha)}
        \end{equation}
        
        \begin{eocproblem*}{1.6}
            For an ideal gas, $PV=nR\theta$ where $n$ is the number of moles. Show that the heat transferred in an infinitesimal quasistatic process of an ideal gas can be written as
            \begin{equation}
                \dbar Q = \frac{C_V}{nR}VdP+\frac{C_P}{nR}PdV.
            \end{equation}
        \end{eocproblem*}
        Let us take the differential of the ideal gas law: $VdP + PdV = nRd\theta$
        For such a process, we have two cases: Constant volume ($dV=0$) and constant pressure $(dP=0)$.
            \begin{equation}
                VdP = nRd\theta \Rightarrow \frac{V}{nR}dP = d\theta
            \end{equation}
            \begin{equation}
                PdV = nRd\theta \Rightarrow \frac{P}{nR}dV = d\theta
            \end{equation}
        Considering the enthalpy.
        \begin{equation}
            H = U + PV \Rightarrow dU + PdV + VdP
        \end{equation}
        Now we have everything we need to solve the question. Let us begin with $dU = \dbar Q + \dbar W = \dbar Q - PdV $.
        \begin{equation}
            \Rightarrow \dbar Q = dU + PdV = \lrp{\periv{U}{\theta}}_Vd\theta + \lrp{\periv{U}{\theta}}_Pd\theta + PdV
        \end{equation}
        Recall the heat capacity at constant volume. Since $dV=0$, $\dbar W=0$ and
        \begin{equation}
            C_V = \lrp{\periv{Q_\text{rev}}{\theta}}_V = \lrp{\periv{U}{\theta}}_V.
        \end{equation}
        Substituting this into (1.5.19),
        \begin{equation}
            \dbar Q = C_Vd\theta + \lrp{\periv{U}{\theta}}_Pd\theta + PdV.
        \end{equation}
        Since we know that the first term has constant volume, we can substitute (1.5.16) for $d\theta$. Similarly, we can multiply and divide the last term by $d\theta$. Then, we have
        \begin{equation}
            \dbar Q = \frac{VC_V}{nR}dP + \lrp{\periv{U}{\theta}}_Pd\theta + P\deriv{V}{\theta}d\theta = \frac{VC_V}{nR}dP + \lrb{\lrp{\periv{U}{\theta}}_P + P\deriv{V}{\theta}}d\theta.
        \end{equation}
        Now, the last term has constant pressure. Thus, we can substitute the expression in (1.5.17). Moreover, note that the term in the paranthesis is the heat capacity at constant pressure. Then, 
        \begin{equation}
            \boxed{\dbar Q = \frac{C_V}{nR}VdP + \frac{C_P}{nR}PdV}.
        \end{equation}

        \begin{eocproblem*}{1.7 from Bowley \& Sanchez}
            Suppose the pressure of a gas equals $\alpha V^{-\gamma}$ for an adiabatic change. Obtain an expression for the work done in a reversible adiabatic change between volumes $V_1$ and $V_2$. 
        \end{eocproblem*}
        
        \begin{equation}
            \dbar W = -PdV + \dbar Q = -\alpha V^{-\gamma}dV
        \end{equation}
        \begin{equation}
            W = -\alpha\int_{V_1}^{V_2}V^{-\gamma}dV = -\frac{\alpha}{1-\gamma}V^{1-\gamma}\bigg|_{V_1}^{V_2}
        \end{equation}
        \begin{equation}
            \boxed{\therefore W = -\frac{\alpha}{1-\gamma}\lrp{V_2^{1-\gamma}-V_1^{1-\gamma}}}
        \end{equation}
\newpage
        \begin{eocproblem*}{1.9 from Bowley \& Sanchez}
            A system consists of a gas inside a cylinder. Inside the cylinder there is a 10$\si{\ohm}$ resistor connected to an electrical circuit through which a current can pass. A mass of $0.1\si{\kilo\gram}$ is placed on the piston which drops by $0.2\si{m}$ and comes to rest. A current of $0.1\si{A}$ is passed through the $10\si{\ohm}$ resistor for $2\si{s}$. What is the change in the internal energy of the system when it has reached equilibrium?
        \end{eocproblem*}
        \begin{equation}
            \Delta U = W = W_m+W_R
        \end{equation}
        \begin{equation}
            W_m = mgh = 0.2\times 0.1 \times 10 = 0.2\si{J}
        \end{equation}
        \begin{equation}
            W_R = Pt = I^2Rt = (0.1)^2\times10\times2=0.1\times2=0.2\si{J}
        \end{equation}
        \begin{equation}
            \boxed{\therefore \Delta U = 0.4\si{J}}
        \end{equation}
        
        \begin{eocproblem*}{1.10 from Bowley \& Sanchez}
            Consider $n$ moles of an ideal monatomic gas at initial pressure $P_1$ and volume $V_1$ which expands reversibly to a new volume $V_2 = V_1(1+\varepsilon)$ with $\varepsilon$ small. Use a $P-V$ diagram to establish in which of the following conditions the system does most and least work; check your answer by calculating the work done by the system in each of these circumstances. [Hint: work to second order in $\varepsilon$]\\
            (a) isobaric (the system expands at constant pressure);\\
            (b) isothermal (the system expands at constant temperature); \\
            (c) adiabatic (no heat enters when the system expands).
        \end{eocproblem*}
        \noindent Starting this problem, we plot the $P-V$ diagrams for each of the cases. First condition has constant pressure throughout. Thus, we have a straight line. In second condition, since the temperature is constant and we have an ideal gas, we can use $PV=nR\theta=$ constant. Then, the relation between the pressure and the volume is $P\propto V^{-1}$. Finally, for the last case recall the result we obtained in (1.3.15) and (1.3.20). For an reversible adiabatic process, we have $P\propto V^{-\gamma}$ and for an ideal gas, $\gamma = 5/3$. Then, the plots are: \\
        \begin{figure}[H]
            \centering
            \begin{tikzpicture}[scale=0.6]
                \begin{axis}[
                    axis lines = middle,
                    xlabel = {$V$},
                    ylabel = {$P$},
                    xmin=0, xmax=3,
                    ymin=0, ymax=2,
                    xtick={1,2},
                    xticklabels={$V_1$, $V_2$},
                    ytick={1},
                    yticklabels={$P_1$},
                    xlabel style={below right},
                    ylabel style={above left}
                ]
                % Isobaric process
                \addplot [
                    domain=1:2,
                    samples=2,
                    color=green,
                    thick,
                    dashed
                ] {1};
                \addplot [
                    domain=0:1,
                    samples=2,
                    color=black,
                    thick,
                    dashed
                ] {1};
                % Isothermal process
                \addplot [
                    domain=1:2,
                    samples=100,
                    color=blue,
                    thick,
                    dotted
                ] {1/x};
            
                % Adiabatic process
                \addplot [
                    domain=1:2,
                    samples=100,
                    color=red,
                    thick,
                    dashdotted
                ] {1/(x^1.4)};
                 
                % Labels
                \node at (axis cs:1.5,1.05) [anchor=south] {isobaric};
                \node at (axis cs:2,0.6) [anchor=west] {isothermal};
                \node at (axis cs:2,0.35) [anchor=west] {adiabatic};
                \end{axis}
            \end{tikzpicture}
            \caption{$P-V$ plots for the three cases}
            \label{fig:prob110}
        \end{figure}
        \noindent Since the work is the are under each of these curves between $V_1$ and $V_2$, we have 
        \begin{equation}
            W_\text{adiabatic} < W_\text{isothermal} < W_\text{isobaric}.
        \end{equation}
        Now, we obtain this result by calculating each value of work by integration. 
        For the first case, since pressure is constant,
        \begin{equation}
            W_\text{isobaric} = P_1(V_2-V_1) = P_1V_1(1+\varepsilon-1) = P_1V_1\varepsilon.
        \end{equation}
        For the second, we use the ideal gas law.
        \begin{equation}
            PV=nR\theta \Rightarrow P = \frac{nR\theta}{V}
        \end{equation}
        \begin{equation}
            W_\text{isothermal} = -\int P dV = -\int_{V_1}^{V_2}\frac{nR\theta}{V}dV = -nR\theta\ln\lrp{\frac{V_2}{V_1}} = -nR\theta\ln\lrp{1+\varepsilon}
        \end{equation}
        Since we want the magnitude, let us take this work as positive. Since $nR\theta$ is constant, we can write it as $P_1V_1$. So that we have a common factor in all three cases. Moreover, let us expand the natural logarithm by its Taylor series up to second order in $\varepsilon$.
        \begin{equation}
            W_\text{isothermal} = P_1V_1\lrp{\varepsilon-\frac{\varepsilon^2}{2}}
        \end{equation}
        Now, we look at the last case. We know that for adiabatic processes, $PV^\gamma = k$ where $k$ is a constant. 
        \begin{equation}
            W_\text{adiabatic} = -k\int_{V_1}^{V_2}V^{-\gamma}dV=-\frac{k}{1-\gamma}\lrp{V_2^{1-\gamma}-V_1^{1-\gamma}}
        \end{equation}
        \begin{equation}
            W_\text{adiabatic} = -\frac{kV_1^{1-\gamma}}{1-\gamma}\lrp{(1+\varepsilon)^{1-\gamma}-1}
        \end{equation}
        Now, we use the binomial expansion to $(1+\varepsilon)$ term. 
        \begin{equation}
            (1+\varepsilon)^{1-\gamma} = 1+(1-\gamma)\varepsilon+\frac{1}{2}(1-\gamma)(-\gamma)\varepsilon^2
        \end{equation}
        Then, 
        \begin{equation}
            W_\text{adiabatic} = -\frac{kV_1^{1-\gamma}}{1-\gamma}\lrb{(1-\gamma)\varepsilon-\frac{\gamma}{2}(1-\gamma)\varepsilon^2}=-kV_1^{1-\gamma}\lrp{\varepsilon-\frac{\gamma}{2}\varepsilon^2}.
        \end{equation}
        Since $k$ is a constant, $P_1V_1^{\gamma}=k$. We again take the magnitude of the work. Then,
        \begin{equation}
            W_\text{adiabatic} = P_1V_1^\gamma V_1^{1-\gamma}\lrp{\varepsilon-\frac{\gamma}{2}\varepsilon^2} = P_1V_1\lrp{\varepsilon-\frac{\gamma}{2}\varepsilon^2}.
        \end{equation}
        Since $\gamma>1$, $W_\text{isothermal}>W_\text{adiabatic}$. And clearly, $W_\text{isobaric}>W_\text{isothermal}$. Thus, we indeed showed that
        \begin{equation}
            W_\text{adiabatic}<W_\text{isothermal}<W_\text{isobaric}.
        \end{equation}